\section{Installation}
\label{sec:inst}
\subsection{Stand-alone}
Assuming you have unpacked the package into the directory \codet{<package\tus dir>} the installation is the following:

\begin{lstlisting}[language=bash]
> cd <package_dir>/make
> make compile
> make link
> make clean
\end{lstlisting}

The first \codet{make} line compiles dynamic library. The second one places it into
standard location (\codet{/usr/lib} in this case); here you probably need root privileges.
The third one deletes object files that are no longer needed.

That's all! At this point you should be able to use the library. However, the following 
step is recommended. It is useful to add package's include to standard C++ include path of 
your system. On Ubuntu adding the following lines in \codet{.profile} (or \codet{.bash\tus profile}) does the work:

\begin{lstlisting}[language=bash]
if [[ -n "$CPLUS_INCLUDE_PATH" ]]; then
    CPLUS_INCLUDE_PATH=$CPLUS_INCLUDE_PATH:/path/to/<package_dir>/inc
else
    export CPLUS_INCLUDE_PATH=/path/to/<package_dir>/inc
fi;
\end{lstlisting}
with \codet{path/to/<package\tus dir>} replaced with the correct path to \codet{<package\tus dir>} directory.

Now logout and login back.

In other Linux distribution you probably would have to use similar procedure. 

\subsection{ROOT-compatible}\label{ssec:install:ROOT}
Read this section only if you intend to use ROOT CERN framework to create \codet{TTree}s
from several binary file. For more details, see Sec.~\ref{ssec:ROOT}.
At this stage you need to install Boost Filesystem Library on your system.

After all the prerequisites are installed the final step is to compile everything into `.so` library: 
\begin{lstlisting}[language=bash]
cd <package_dir>/make
root -l CompileROOT.C
\end{lstlisting}
