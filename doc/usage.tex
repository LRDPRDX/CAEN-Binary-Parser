\newpage
\section{Usage}
\subsection{Stand-alone}
\subsubsection{Entire structure}\label{ssec:ent_str}
In order to prevent name conflicts and keep global scope clean everything is placed into
namespace \codet{caen}.

This chapter covers two forms of usage:
\begin{itemize}
    \item as a stand-alone pure C++ software
    \item as a part of the software based on the \textbf{ROOT CERN} framework
\end{itemize}

\Note{In the below code the \codet{using} declaration is \emph{not}
assumed so each entity of the library appears with the \codet{caen::} prefix.}

So far there are three main classes on the scene. Those are: \codet{Event}, \codet{Parser},
\codet{Analyzer}.
The diagram on Fig.\ref{fig:diag} illustrates how they interact. 

\tikzstyle{emptyclass}=[rectangle, draw=black, rounded corners, fill=white, drop shadow,
        text centered, anchor=north, text=black, text width=3cm]
\tikzstyle{fullclass}=[rectangle, draw=black, rounded corners, fill=blue!20, drop shadow,
        text centered, anchor=north, text=black, text width=3cm]
\tikzstyle{class}=[rectangle, draw=black, rounded corners, fill=violet!30, drop shadow,
        text centered, anchor=north, text=white, text width=4cm]
\tikzstyle{comment}=[rectangle, draw=black, rounded corners, fill=green, drop shadow,
        text centered, anchor=north, text=white, text width=3cm]
\tikzstyle{myarrow}=[->, >=triangle 90, thick]
\tikzstyle{line}=[-, thick]
        
\begin{figure}[H]
\begin{tikzpicture}[node distance=3cm]
    \node (EmptyEvent) [emptyclass, rectangle split, rectangle split parts=2]
        {
            \codet{Event} 
            \nodepart{second}
                data is empty
        };
    \node (Parser) [class, rectangle split, text width=4.5cm, rectangle split parts=2, right=2cm of EmptyEvent]
        {
            \codet{Parser} 
            \nodepart{second}
                \begin{itemize}
                    \item[+] \codet{ReadEvent( Event\& )}                 
                    \item[+] ...
                \end{itemize}
        };
    \node (FullEvent) [fullclass, rectangle split, rectangle split parts=2, right=2cm of Parser]
        {
            \codet{Event} 
            \nodepart{second}
                data is filled 
                \begin{itemize}
                    \item[+] \codet{cbegin()}
                    \item[+] \codet{cend()}
                    \item[+] \codet{GetSize()}
                    \item[+] ...
                \end{itemize}
        };
    \node (Aux) [text width=3cm, below of=FullEvent] {};
    \node (EmptyAnalyzer) [emptyclass, text width=5cm, rectangle split, rectangle split parts=2, left=2cm of Aux]
        {
            \codet{Analyzer} 
            \nodepart{second}
                data is empty
                \begin{itemize}
                    \item[+] \codet{Analyze( const Event\& )}
                \end{itemize}
        };
    \node (FullAnalyzer) [fullclass, text width=4cm, rectangle split, rectangle split parts=2, left=2cm of EmptyAnalyzer]
        {
            \codet{Analyzer} 
            \nodepart{second}
                data is filled 
                \begin{itemize}
                    \item[+] \codet{GetBaseline()}
                    \item[+] \codet{GetIntegral()}
                    \item[+] ...
                \end{itemize}
        };
    \draw[myarrow] (EmptyEvent.east) -- node[above]{passed by} node[below]{reference} (Parser.west);
    \draw[myarrow] (Parser.east) -- (FullEvent.west);
    \draw[] (FullEvent.south) -- (Aux.center);
    \draw[myarrow] (Aux.center) -- node[above]{passed by} node[below]{reference} (EmptyAnalyzer.east);
    \draw[myarrow] (EmptyAnalyzer.west) -- (FullAnalyzer.east);
\end{tikzpicture}
\caption{Interaction between classes during processing an event.}
\label{fig:diag}
\end{figure}


In this section one will find a brief description of the abovementioned classes and how to work with them.
For more details see sec~\ref{sec:ref}.

\paragraph*{\codet{Event} class.}

The \codet{Event} class represents a single event. It serves to store primary information
of an event: its size, trigger time tag, ..., and data points. Also it has some additional 
member functions that might be useful (see sec.\ref{sec:ref}).

One usually needs only one instance of this class (see below).

\paragraph*{\codet{Parser} class.}

The \codet{Parser} class is designed for event construction. More precisely it takes
an empty event and \emph{fill} it according to the data in a binary file produced by
\blt{WD}.

\paragraph*{\codet{Analyzer} class.}

The \codet{Analyzer} class is designed for (no surprise) performing simple analysis on the
event data. Unlike the \codet{Parser} class this class deals with events which are already 
constructed and filled by the \codet{Parser} object.
\codet{Analyzer} \emph{doesn't change an event}. In other words, events are read-only for
\codet{Analyzer}. 

\paragraph*{\codet{Board} class.}
There is one more class which must be mentioned. It is the \codet{Board} class. It
represents a digitizer's model and serves as a translator between abstract
digital and physical scales. For example, the same waveform may be of different length
(having the same number of points) depending on a digitizer which produced it because
the length (or sample time) depends on the sample rate. 

\subsubsection{Read an event}
In order to be able to read an event from \blt{WD}'s binary file user should follow these
steps: 
\begin{enumerate}
    \item Choose correct digitizer's model 
    \item Create \codet{Parser} object
    \item Set path to file to parse using \codet{Parser::SetPathToFile} member function
    \item Create \codet{Event} object
    \item Call \codet{Parser::ReadEvent} or \codet{Parser::ReadEventAt} member function
with previously created \codet{Event} object as an argument
    \item Use \codet{Event}'s member functions to retrieve information about the read
event
    \captionof{algorithm}{Reading an event from a binary}
\end{enumerate}

\noindent Let us describe each step in detail below.

\paragraph*{Choosing digitizer's model.}
In order to select digitizer's model user has to pass object of \codet{Board} class
as an argument to the constructor of \codet{Parser} class (see below). \codet{Board}
class can be instantiated using either of two constructors. The first one allows
user to choose a model from a list of known digitizers
(constructed from \href{https://www.caen.it/subfamilies/digitizers/}{this} list).
The second one is used when the desired model is not present in the list. For more
details see sec.\ref{sec:ref}.

\begin{lstlisting}
#include "CaenParser.h"//CaenBoard.h header included here

caen::Board boardOne( caen::Board::N6720 );//using built-in enum
caen::Board boardTwo( 12, 250000000, 2.0 );//providing board's characteristics by hand 
\end{lstlisting}

\Note{If you didn't set the include path as it was recommended in Sec.\ref{sec:inst} you
would probably have to provide full path in the \codet{include} directive. However, it is
not the case if you specify the include path (with \codet{-I} option) when compiling.}

\paragraph*{Creating \codet{Parser} object.}
\codet{Parser} class has the only constructor that takes \codet{Board} object as an
argument:
\begin{lstlisting}
#include "CaenParser.h"

caen::Parser parser( caen::Board::N6720 );//note copy-initialization for Board class
\end{lstlisting}

\paragraph*{Setting path to binary.}
Once you have chosen the board's model it's time to specify
file to work with. Its type must be \codet{std::string}. Most probably the
name of such file would be \codet{waveN.dat} (unless has been changed by user) where
\codet{N} is channel number (starts
from 0):
\begin{lstlisting}
parser.SetPathToFile( "path/to/waveN.dat" );//note extension   
\end{lstlisting}
The path may be either absolute or relative.

\paragraph*{Creating \codet{Event} object.}
The \codet{Event} class has the only constructor that takes no arguments:
\begin{lstlisting}
#include "CaenEvent.h"

caen::Event event;
\end{lstlisting}

As it was mentioned in Sec.\ref{ssec:ent_str} one usually needs only one instance of
each \codet{Event} and \codet{Parser} class. But there are cases when several instances
are useful (or even necessary). For example, in case when comparing several channels
is required. As far as \blt{WD} produces separate data file per channel one should have
its own \codet{Parser} instance for each channel. And as it will be clear from the below
text in this case one should also use its own \codet{Event} instance for each channel.

\subsubsection*{Reading an event}
Next step is to read an event from a binary file produced by \blt{WD}. To do this user
should call \codet{ReadEvent} member function with \codet{Event} object as an argument
(passed by reference):

\begin{lstlisting}
parser.ReadEvent( event );
//OR
parser.ReadEventAt( 0, event );
\end{lstlisting}

\codet{ReadEvent} member function returns \codet{true} if the event has been
successfully read and \codet{false} otherwise. Here are some steps that
\codet{ReadEvent} member function performs:

\begin{enumerate}
\item Reset current event
\item
    \begin{itemize}
        \item (\codet{ReadEvent}) Try to read and fill the event from the position
followed after the end of previous event --- current position indicator (or from
the beginning of the file in case when no event hasn't been read yet)
        \item (\codet{ReadEventAt}) Try to read and fill the event from the
specified position in binary file 
    \end{itemize}
\item (\codet{ReadEvent} only) If 2. succeeded then update position indicator
\end{enumerate}

The above algorithm is not exactly what \codet{ReadEvent} does but here are several things 
to note. First of all, it \emph{resets the current event}. This means that at the moment
of invocation of \codet{ReadEvent} member function a variable that was passed to it
\emph{is no longer stores whatever it did before the invocation} even if reading failed. The second thing to
note is that \codet{ReadEvent} member function reads one event at a time. You could only
read events one-by-one, one after another. Every time you want to read new event you must
invoke \codet{ReadEvent} member function.


Eventually the complete program may look like this: 
\begin{lstlisting}[numbers=left, caption={Print size of each event stored in \codet{waveN.dat} file}]
#include "CaenParser.h"
#include "CaenEvent.h"


int main()
{
    caen::Parser parser( caen::Board::N6720 );
        parser.SetPathToFile( "path/to/waveN.dat" );
    caen::Event event;

    while( parser.ReadEvent( event ) )//read event by event
    {
        //Information about the current event is accessible
        //through Event's member functions 
        std::cout << event.GetSize() << "\n";
    }

    return 0;
}
\end{lstlisting}

\subsubsection{Analyse an event}\label{sssec:anal}
Often having just primary data is not enough. One may need some property of a signal that
is not in the data file. It might be integral over a given time interval or time point at the maximum
displacement from the baseline. Those purposes are the \codet{Analyzer} class was written
for. It is defined in \codet{CaenAnalyzer.h} file. This class is instantiated by calling the only
constructor that takes no arguments:

\begin{lstlisting}
#include "CaenAnalyzer.h"

caen::Analyzer analyzer;
\end{lstlisting}

In order to perform analysis of an event user calls correspondig mamber function of this
class with the desired \codet{Event} object as an argument (passed by reference):
\begin{lstlisting}
analyzer.Analyze( event );
\end{lstlisting}
\paragraph*{Analysis config.}
However the above line is not enough for complete analysis. User should provide the
\emph{analysis config}. Some of the calculations use the user-defined parameters. For example, if
you want to get integral over time of a signal you should specify time range of
integration. Moreover, in order to calculate integral it is required to know the baseline
(or zero level). So far \codet{Analyzer} has two simple methods to calculate zero level.
For both of them user should also specify time range (see Sec.\ref{sec:ref}). So the
analysis config consists of the following settings (the order is not of importance):
\begin{itemize}
    \item{Method to calculate baseline. Now there are two methods (see sec.\ref{sec:ref}).
Member function to use: \codet{SetBaselineMethod( Analyzer::BASELINE method )}}
    \item{Time interval for baseline calculation. It starts at the beginning of a signal
and ends at the time defined by user. Member function to use: \codet{SetBaselineInterval( double T )}}
    \item{Time interval for integral calculation. It starts and ends at time points defined
by user. Member function to use: \codet{SetGate( double start, double end )}}
\end{itemize}

\Note{The analysis config should be done \emph{before} the invocation of \codet{Analyze}
member function.}

\noindent So the complete program with analysis part usually contains the following lines:
\begin{lstlisting}[numbers=left, caption={Print integral of each event in \codet{waveN.dat} calculated within the range from 40 to 100 nanosecs}]
#include "CaenParser.h"
#include "CaenAnalyzer.h"
#include "CaenEvent.h"


int main()
{
    caen::Parser parser( caen::Board::N6720 );
        parser.SetPathToFile( "path/to/waveN.dat" );
    caen::Event event;
    caen::Analyzer analyzer;

    //Analysis config
    analyzer.SetBaselineInterval( 0.03 );//set range for baseline calcultation: the first 30 ns of the signal
    analyzer.SetGate( 0.04, 0.1 );//set integral range between 40 and 100 ns

    while( parser.ReadEvent( event ) )
    {
        //Perform analysis
        analyzer.Analyze( event );
        //Now the results are available through the getters 
        std::cout << analyzer.GetIntegral() << "\n";
    }

    return 0;
}

\end{lstlisting}

\subsubsection{Compilation}
Provided you put your code in the file named \codet{example.cpp} the following line should compile
it into \codet{example} executable:
\begin{lstlisting}
g++ example.cpp -std=c++11 -I<path/to/package_dir>/inc -lcaenparse -o example 
\end{lstlisting}
It is more convenient to put this in a \codet{Makefile} (one may find an example of such
\codet{Makefile} in \codet{example/stand\tus alone} directory).

\subsection{ROOT-compatible}\label{ssec:ROOT}
In this section it will be explained how to use this library together with the ROOT CERN
framework, in particular, how to create a \codet{TTree} from several binary files. Provided
you have followed the instructions in Sec.~\ref{ssec:install:ROOT} the following algorithm
should be used to create a \codet{TTree} from a set of a binary files:
\begin{enumerate}
    \item Create an instance of the \codet{CaenTreeParser} class using the same \codet{Board}
        as was used to produce the binary
    \item Set the analysis config using member-functions of the \codet{CaenTreeCreator} class
    \item Call the \codet{CreateTree} member-function
    \item Compile and run
    \label{algo:tree}
    \captionof{algorithm}{Creation of a \codet{TTree}}
\end{enumerate}
Below is the explanation of the above algorithm in detail. 

\paragraph*{The \codet{CaenTreeCreator} class.} This class is responsible for
creation of a \codet{TTree} from a data binary file.
To create an instance of this class you, firstly, include its header:
\begin{lstlisting}
#include "ROOT/CaenTreeCreator.h"
\end{lstlisting}
Then you use the only constructor of this class which takes four arguments:
\begin{lstlisting}
TreeCreator(   caen::Board board,
               const std::string& pathToDataDir,
               const std::string& pathToTreeFile,
               const std::string& treeFileName )
\end{lstlisting}
where
    \begin{itemize}
        \item[] \codet{board} --- board model which was used to record data 
        \item[] \codet{pathToDataDir} --- path to the directory containing the binaries (see the NOTE below)
        \item[] \codet{pathToTreeFile} --- path to the location where the resulting \codet{TTree} will be placed
        \item[] \codet{treeFileName} --- name of a \codet{.root} file (without an extension) with the resulting \codet{TTree}
    \end{itemize}

The meaning of the second argument requires additional explanation. There may be two
cases here. The first is the one when the target directory contains no other directories
(subdirectories) --- only binary files. I.e. the structure is the following:
\begin{lstlisting}[language=bash]
path
|
|_to
  |
  |_data <--target
     |
     |__wave01.dat
     |__wave02.dat
     |__...
\end{lstlisting}
In this case, provided the second argument in the \codet{CaenTreeCreator} constructor
was \codet{"path/to/data"}, THE ONLY \codet{TTree} named \codet{tree\tus} would be constructed
from all the binaries in the \codet{data} directory. If, on the other hand, the target
directory contains other directories, for example:
\begin{lstlisting}[language=bash]
path
|
|_to
  |
  |_data <--target
    |
    |_data1
    |   | 
    |   |__wave01.dat
    |   |__wave02.dat
    |   |__...
    |
    |_data2
      |
      |__wave01.dat
      |__wave01.dat
      |__...
\end{lstlisting}
then, provided the second argument in the \codet{CaenTreeCreator} constructor
was \codet{"path/to/data"}, TWO \codet{TTree}s named \codet{tree\tus data1} and
\codet{tree\tus data2} would be constructed from all the binaries in the \codet{data1} and
\codet{data2} directory (and in its subdirectories), respectively.
It means that hierarchy of the subdirectories of the target directory doesn't matter:
all the binaries are searched recursively in a subdirectory of the target directory. I.e.
the \codet{subsubdata} is expanded when constructing a \codet{TTree} provided the following
structure:
\begin{lstlisting}[language=bash]
path
|
|_to
  |
  |_data <--target
    |
    |_data1 <-- matters
    |   | 
    |   |__wave01.dat
    |   |__wave02.dat
    |   |__...
    |
    |_data2 <--hierarchy matters
      |
      |__wave01.dat
      |__wave02.dat
      |__...
      |__subsubdata <-- hierarchy DOESN'T matter: expanded
         |__wave01.dat
         |__wave02.dat
         |__...
\end{lstlisting}
However, the path (see below) of the every binary is conserved in the resulting
\codet{TTree}.

After the instantiating the \codet{CaenTreeCreator} you should set the analysis config
(see Sec.~\ref{sssec:anal}) which will be used when constructing the \codet{TTree}:
\begin{lstlisting}
    tc.SetIntervals( 0.1, 0.05, 0.2 );//baseline - from the beginning to 100 ns; integral - from 50 to 200 ns
\end{lstlisting}

\Note{Setting the analysis config for the \codet{TreeCreator} class differs slightly from that for the \codet{analyzer} class (see~Sec.\ref{ssec:ref:tree})}

After that you call the \codet{CreateTree} member-function. The whole code could be:

\begin{lstlisting}
#include "ROOT/CaenTreeCreator.h"
#include "CaenParser.h"


void CreateTree()
{
    CaenTreeCreator tc( caen::Board::N6720,               //N6720 ADC was used for recording
                    , "./Data", "./", "myFirstTree" );
    tc.SetIntervals( 0.03, 0.03, 0.1 ); //baseline - from 0 to 30 ns; integral - from 30 to 100 ns 

    tc.CreateTree();
}
\end{lstlisting}

\paragraph*{Compile and run.} To run the above example create (or append) the \codet{rootlogon.C} file in the directory
where you put this example with the following line:
\begin{lstlisting}
{
    gSystem->AddLinkedLibs( "path/to/<package_dir>/src/ROOT/CaenTreeCreator/CaenTreeCreator_cpp.so" );
}
\end{lstlisting}
where \codet{/path/to/<package\tus dir>} is the absolute path to the \codet{<package\tus dir>}.

Then type the following in your working directory (it is assumed you have saved the code in the file \codet{CreateTree.C}):
\begin{lstlisting}
root -l CreateTree.C+
\end{lstlisting}
After that there will appear the file (along with the others) named \codet{myFirstTree.root} in the current directory with a \codet{TTree}. The full example (with the demo data)
see in the \codet{<package\tus dir>/example/ROOT/create\tus tree} directory.
